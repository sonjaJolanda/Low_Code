\documentclass{article}
\usepackage[T1]{fontenc}
\renewcommand*\contentsname{Inhalt}

\usepackage[backend=biber]{biblatex}
\addbibresource{Low_Code_Citavi.bib}

\title{Werden sich Low-Code Development Platforms durchsetzen?}
\author{Sonja Klein}
\date{10.01.2023}

% begin of the document
\begin{document}
	
	Hier sollte ich das Abstract rein schreiben
	
	\maketitle	
	\tableofcontents	
	\newpage
	
	\section{Einleitung}
	
	Low Code Development Plattformen sind benutzerfreundliche Umgebungen, die vor allem bei der Entwicklung mobiler Anwendungen zunehmend an Bedeutung gewinnen. In jüngster Zeit werden sie zunehmend von großen IT-Unternehmen eingeführt und gefördert, um die Entwicklung von Softwareanwendungen zu beschleunigen und den typischen Zeitdruck in verschiedenen Bereichen zu bewältigen. Bis zu einem gewissen Grad könnte Low-Code-Engineering als ein Synonym und eine Weiterentwicklung des traditionellen Model-Driven-Engineering (MDE) betrachtet werden.
	Bei Low-Code-Entwicklungsplattformen erstellen die Benutzer "Modelle", um die Schlüssellogik zu konzipieren, und der Code wird automatisch generiert.
	\cite{Wang.2021}
	
	Der wichtigste Treiber der digitalen Transformation ist die Verbesserung der bestehenden IT-Fähigkeiten - und Unternehmen, die Low-Code für Top-Anwendungen nutzen, geben dies als den häufigsten Grund für die Nutzung von Low-Code-Plattformen (53 \%) an.
	Sie berichten auch von größerer Agilität (43 \%) und geringeren Kosten (42 \%). Es ist also nicht überraschend, dass 95 \% dieser Firmen angeben, sie seien zufrieden oder sehr zufrieden mit den unternehmensorientierten Features ihrer Low-Code-Plattformen. 
	Da 72 \% aller Unternehmen Out-of-the-Box-Anwendungen mit individuellem Code anpassen und 57 \% vollständig benutzerdefinierten Code verwenden um Top-Anwendungen bereitzustellen, existiert erhebliches Potenzial für Low-Code-Plattformen einen größeren Anteil an der digitalen Agenda einzunehmen. \cite{EmmaVanPelt.2019}
	
	Aus der kontinuierlichen Ausweitung der Softwareentwicklung für künstliche Intelligenz (KI), das Internet der Dinge (IoT), Robotik und andere Automatisierungsanwendungen ergibt sich eine steigende Nachfrage nach Softwareentwicklern, Analysten für Software-Qualitätssicherung und Testern.
	Die Gesamtbeschäftigung von Softwareentwicklern, Qualitätssicherungsanalysten und Testern wird von 2021 bis 2031 voraussichtlich um 25 Prozent zunehmen und damit deutlich schneller steigen als der Durchschnitt aller Berufe.
	% https://www.bls.gov/ooh/computer-and-information-technology/software-developers.htm#tab-6
	
	Um diesem Mangel entgegenzuwirken und den steigenden Bedarf an IT-Lösungen zu befriedigen, sollte eine zeitgemäße Software- und Prozessentwicklung einfach und effizient sein und auch weniger qualifizierte Mitarbeiter (in Bezug auf ihre Programmierkenntnisse) an IT-Entwicklungsaufgaben beteiligen (Richardson und Rymer 2014).
	%Trotzdem gab es im Jahr 2019 124.000 offene IT-Stellen in der deutschen Wirtschaft, so Bitkom Research (2019), was einen 	Anstieg von 51 \% im Vergleich zum Vorjahr bedeutet. Fast ein Drittel davon waren Softwareentwickler (Bitkom Research 2019). 
	%Internationale Studien zum Mangel an IT-Fachkräften zeigen ähnliche Ergebnisse (Cushing 2019; Harvey Nash und KPMG 2018).
	
	\subsection{Relevanz und Ziel der Arbeit}
	
	%Software-Projekte haben keine so mega hohe Erfolgsquote, LCDPs haben den Anspruch das zu verbessern <- Quellen finden und ausformulieren.
	%vielleicht auch noch die NASCIO Studie zitieren, die belegt dass Unternehmen Low Code als wichtigen Trend betrachten
	
	%--https://kar.kent.ac.uk/89629/1/JBR_ManuscripAugust2021R&R3final%20REF.pdf <- das für die Relevanz (Corona hat die Digitalisierung voran getrieben und deshalb brauchen wir auch Low Code!) <- Rückwärtssuche Niculin
	
	Die digitale Welt ist die Plattform geworden auf der Kunden gewonnen und verloren worden. Um wettbewerbsfähig zu bleiben sind Unternehmen deshalb gezwungen zu digitalisieren. Auf dem Weg der digitalen Transformation stoßen Unternehmen jedoch an ihre Software-Entwicklungs-Grenzen - sei es im Talente Einstellen und Halten, beim interdisziplinären %(?! also zwischen ) Entwickner und Fachkundigen %	
	oder bei der Lösungsbereitstellung mit wettbewerbsfähiger Markteinführungszeit.	
	Um diesen Grenzen entgegen zu wirken, hat sich in den letzten Jahren ein neuer Trend entwickelt: Low Code Entwicklungsplattformen. \cite{EmmaVanPelt.2019} %- they said: Unauthorized reproduction is 	strictly prohibited. Information is based on best available resources.
	
	Appian beauftragte Forrester Consulting im Dezember 2018 damit, herauszufinden, ob XXXX. Forrester führte eine Online-Umfrage unter 254 IT- und Line-of-Business Entscheidungsträgern in den USA, Großbritannien, Kanada und Australien durch, um Erwartungen an und Erfahrungen mit  Low-Code-Entwicklungsplattformen für Unternehmensanwendungen zu bewerten. Dabei fanden sie heraus, dass zwar einige Entscheidungsträger bezweifeln, dass Low-Code-Plattformen unternehmensweit unterstützen können, Firmen mit den höchsten Unternehmensanforderungen aber erfolgreich kritische Anwendungen mit Low-Code-Plattformen betreiben.	  \cite{EmmaVanPelt.2019}	
	

	Low Code erlebte einen Aufschwung während der Pandemie. Neben Teams und Zoom wuchs auch die Relevanz von Low-Code Development Plattformen. Die außergewöhnlichen Umstände brachten noch nie dagewesene und dringende Challenges mit sich und Low-Code war oftmals die Lösung auf diese. Corona-Testzentren brauchten innerhalb kürzester Zeit Terminportale und mit Low-Code konnten diese in bis zu zwei Tagen entwickelt werden.
	\cite{AmyGlasscock.2021}
	\newline \newline 
	LC/NC gewinnt immer mehr an Bedeutung, das wird auch in einer staatlichen Umfrage von NASCIO deutlich. 2020 und 2021 wurden CIOs gefragt, welche neue Technologie ihrer Meinung nach in den nächsten 3 bis 5 Jahren die größte Bedeutung haben wird. Während 2020 LC/NC noch auf dem zweiten Platz mit 33\%, direkt hinter AI (Künstlicher Intelligenz, RPA, ...) mit 61\% lag, teilte LC/NC sich 2021 schon den ersten Platz mit 31\%. AI machte in diesem Jahr nur noch 30\% aus.
	\cite{AmyGlasscock.2021}
	
	\subsection{Methodik}
	Zur Beantwortung der Fragestellung dieses Textes wird die Methodik der Literaturrecheche verwendet.
	Zu Beginn wurde erst einmal ein grundlegendes Verständnis der Fragestellung und der Begrifflichkeiten durch das Internet und YouTube geschaffen. %kann und soll ich das so schreiben??
	 
	Als nächstes wurde ein erster Eindruck über die Quellenlage geschaffen. In den online-Bibliotheken der Hochschule für Technik, Wirtschaft und Gestaltung Konstanz und in der Universität Konstanz (KonSearch) zum Thema Low Code war nicht viel zu finden. Deshalb musste auf andere Quellen ausgewichen werden. \newline
	Die Suche in Google Scholar, SpringerLink und AISnet.org brachten einige relevante Quellen. 
	\newline
	Und auch zwei Paper von Niculin Prinz (und deren Rückwärtssuche) erweiterte die Quellenlage. %soll ich das schreiben?
	
	In der Suche lag der Fokus auf folgenden Begrifflichkeiten: "", "", "" und "".
	
	%Basierend auf den ersten Ergebnissen der Stichwortsuchmethode und einer anschließenden Rückwärtssuche konnte eine Vielzahl von Quellen, die im Literaturverzeichnis aufgelistet, ausgearbeitet werden.	
	
	\subsection{Aufbau der Arbeit}
	Die vorliegende Arbeit ist in X Kapitel untergliedert. Zunächst wird dem Leeser im aktuellen Kapitel die Struktur, der Ablauf, das Verfahren, sowie die Wichtigkeit der Hausarbeit erläutert. 
	
	In dem zweiten Kapitel wird erstmalig an die Fragestellung mit der Entstehung und Definition von Low Code Development Platforms herangeführt. 
	
	%bei Daorsa Formulierungen abschauen
	
	Der Abschluss der Arbeit umfasst ein Fazit, welches die Beantwortung der Leitfrage "" zum Ziel hat.
	
	%---------------------------------------------------------------------------------------------------
	\section{Entstehung von LCDP}
	In diesem Kapitel wird zunächst auf die Entstehung von Low Code Entwicklungsplattformen eingegangen. Daraufhin werden grundlegende Begriffe der Arbeit definiert. 
	%also Low Code, High Code, No Code, Citicen Developer, ...
	
	\subsection{Entstehung}	
	Die Mainstream-Hochsprachen der Programmierung (High-Level im Vergleich zu Assembler Sprachen und Maschinencode) haben sich seit dem Aufkommen der Sprache Fortran vor gut einem halben Jahrhundert dramatisch weiterentwickelt, wobei Hunderte von Sprachen seitdem entwickelt wurden.\cite{Margaria.2021}
	%- 
	
	%Das erste Mal kam das Thema 1982 auf:  James Martin  "Application Development Without Programmers" veröffentlicht". Er sagt dort dass dadurch das Computer immer billiger werden, Computer auch billiger werden als die Ressource Mensch  und es deshalb unausweichlich ist, dass in der Zukunft immer mehr Computer mit weniger Programmierern zusammengesetzt werden müssen.
	
	%Wie bei den meisten 4GL-( 4th-generation prog. language )  und visuelle Prog.technologien waren zwar ein großer Schritt für die IT, aber die Tools selbst konnten dem Hype einfach nicht gerecht werden. Besonders schwierig war es, Anwendungen zu entwickeln, die sich skalieren ließen.
	
	%Die Tools unterstützten keine Best Practices. Versionskontrolle, Tests, Bereitstellung, Dokumentation und andere Best Practices für die Entwicklung mussten manuell durchgeführt werden.
	
	%LCDPs verstärkten die Sicherheitsrisiken. Die Beauftragung von Citizen Devs mit der Entwicklung brachte die Tatsache mit sich, dass Cit.Devs nicht über die erforderlichen Fähigkeiten verfügten, um Anwendungen mit angemessener Sicherheit und Governance zu erstellen und bereitzustellen.
	
	%Das Internet schluckte alles. Mitte der 2000er Jahre konzentrierte sich bereits ein erheblicher Teil der SWEN auf Webanw., da immer mehr U die Produktivität ihrer Mitarbeiter steigern wollten, indem sie Geschäftsanw. über die Cloud statt über traditionelle Serverumgebungen bereitstellten. Dadurch wurde ein Teil des Bedarfs an traditionellen IT-Lösungen für alltägliche Probleme ausgeglichen.
	
	%Idee gut aber Zeitpunkt nicht gepasst, die Technologien waren noch nicht ausgereift.	
	
	%Mit der Zeit wurden die Prog.sprachen immer fortschrittlicher + immer mehr darauf ausgelegt das Programmieren möglichst effizient zu machen. Außerdem bietet  die neue Generation von LCDP keine Schnittstelle, die den eigentlichen Code einer Anwendung verdeckt, sondern in sich geschlossene Plattformen, die es den Anwendern ermöglichen, in einer Umgebung zu entwickeln, die bereits alle unsichtbaren Komponenten der Anwendung enthält. Die meisten modernen LCAP werden über das Internet bereitgestellt, so dass sich die Benutzer nicht um Updates kümmern müssen.
	
	% Der Cloud-Plattform-Ansatz ermöglicht es diesen Tools auch, weitaus mehr Sicherheit und Zuverlässigkeit als je zuvor zu bieten. So können U viel einfacher darauf vertrauen, dass sie über die richtigen Kontrollen verfügen, um ihre Sicherheits- und Compliance-Standards zu erfüllen. Wenn die Plattform selbst ein hohes Maß an Sicherheits-Kontrollen bietet, ist der Weg zur sicheren Bereitstellung von Plattformanwendungen wesentlich kürzer.
	
	% Schließlich ist die Benutzerbasis für diese Plattformen in den letzten zehn Jahren erheblich gereift, da die erfolgreichsten U der Welt LCDP für ihre einzigartigen Prozesse nutzen, was zu Best Practices, einem florierenden Ökosystem von Partnern und LC-Entwicklern und einem insgesamt besseren Verständnis der Fähigkeiten jeder Plattform geführt hat.
	
	\subsection{Definition und Begriffsabgrenzungen}
	%da vielleicht mehrere Subsections machen mit Low Code, Hight code, Citizen Developer usw (oder vielleicht ist das auch zu wenig für subsections)
	
	%LCDP ist eine Familie von Entwicklungstools. Also jede hat andere Merkmale.
	% LCDP vereinfachen die Entw. indem sie Techniken wie Drag-and-Drop Funktionalitäten + visuelle Tools zur Entwicklung bereitstellen.
	% Man versucht die manuelle Codierung so weit wie möglich zu reduzieren. 
	% Dadurch kann jeder unabhängig von seinen techn. Fähigkeiten und Kenntnissen entwickeln.  
	
	% Allerdings sind nicht alle Geschäftsprobleme für LC geeignet: 	Einfache, geschäftsorientierte Projekte sind ein guter Ausgangspunkt. Eigentlich muss man einfach bevor man sich für den Low Code Ansatz entscheidet prüfen ob es für den eigenen Anwendungsfall eine geeignete LCDP gibt, denn diese werden immer funktionalitäten-reicher.
	
	%-----------------------------------------------------
	Diese befähigten Mitarbeiter werden als Bürgerentwickler bezeichnet und sind hauptsächlich entweder Power-User, Entwickler in einer Fachabteilung oder reguläre Mitarbeiter in der Fachabteilung (McKendrick 2017).
	IT-Entwicklungsplattformen im Unternehmen helfen ihnen, Geschäftsanwendungen oder Workflows unabhängig von der IT-Abteilung des Unternehmens zu entwickeln (Rollings 2012). 
	Um einen Hypernamen für diese Plattformen zu etablieren, prägte Forrester Research (Richardson und Rymer 2014) erstmals den Begriff "Low-Code Development Platform" (LCDP) im Jahr 2014. Die Autoren charakterisieren LCDPs als eine enorme Reduzierung von Handcodierung, als schnellere Bereitstellung von Anwendungen mit Hilfe von visuellen Tools und als die Fähigkeit, Daten effektiv aufzubereiten, um mehrstufige Workflows zu erstellen. 
	Eine weitere Veröffentlichung (Tisi et al. 2019) definiert sie als Software-Entwicklungsplattform in der Cloud, die ein Platform-as-a-Service (PaaS)-Modell bietet, mit dem Nutzer schlüsselfertige betriebliche Anwendungen mit deklarativen Sprachen, dynamischen grafischen Benutzeroberflächen (UI) und visuellen Diagrammen.
	%----das war alles von Niculin
	
	\subsection{Beispiele}	
		%- marktführer an beispielen sind irgendwie falsch also in marktstudien nennen -> richtige finden
	Lorem  ipsum  dolor  sit  amet,  consectetuer  adipiscing  
	elit.   Etiam  lobortisfacilisis sem.  Nullam nec mi et 
	neque pharetra sollicitudin.  Praesent imperdietmi nec ante. 
	
	\subsection{Use Cases}	
	% MVPs: Ideen auszuprobieren geht mit LCDP schneller und billiger denn je. 
	
	% neue Plattformen: wie Augmented Reality, Virtual Reality oder dialogorientierte Schnittstellen. Einige Multiexperience DPs (MXDPs) ermöglichen die Entw. für diese über NC/LC 
	
	% Smart Process Apps: Operational efficiency apps sind Anwendungen mit dem Ziel durch die Automatisierung manueller Prozesse Kosten zu senken. Die North Carolina State Uni bspw. nutzt eine LCDP, um eine App zur Kursanmeldung zu erstellen, die 500.000 Anmeldungen für Nicht-Kreditkurse pro Jahr ermöglicht.
	
	% Modernisierungen: Legacy-Migrationsanwendungen zielen darauf ab, Anwendungen zu ändern, die keine neuen Prozesse unterstützen oder die richtige Benutzererfahrung bieten können.
	 
	Lorem  ipsum  dolor  sit  amet,  consectetuer  adipiscing  
	elit.   Etiam  lobortisfacilisis sem.  Nullam nec mi et 
	neque pharetra sollicitudin.  Praesent imperdietmi nec ante.
	
	
	\subsection{Anwendungsfall: MVPs} 
	
	%---------------------------------------------------------------------------------------------------
	\section{Nachteile von LCDP}
	
	In diesem Kapitel werden unterschiedliche Nachteile von Low Code Development Platforms genauer unter die Lupe genommen. Neben einer Erläuterung des Nachteils wird auch die Relevanz jedes Nachteils für Unternehmen genauer betrachtet. %MACHEN!!!!
		
	\subsection{geringe Skalierbarkeit}	
	Lorem  ipsum  dolor  sit  amet,  consectetuer  adipiscing  
	elit.   Etiam  lobortisfacilisis sem.  Nullam nec mi et 
	neque pharetra sollicitudin.  Praesent imperdietmi nec ante.
	
	\subsection{Sicherheitsrisiken}	
	Lorem  ipsum  dolor  sit  amet,  consectetuer  adipiscing  
	elit.   Etiam  lobortisfacilisis sem.  Nullam nec mi et 
	neque pharetra sollicitudin.  Praesent imperdietmi nec ante.  
	
	\subsection{geringere Flexibilität und Anpassbarkeit}	
	Ein Nachteil wen Low Code Plattformen mit sich bringen ist die geringe Flexibilität und Anpassbarkeit der entwickelten Anwendung. %Erklärung und Quelle
	
	%---------------------------------------------------------------------------------------------------
	\section{Potenziale und Vorteile von LCDP}
	
	Bei vielen Entwicklern wird beim LC Ansatz aber noch gezögert. 
	%Quelle (vielleicht in EmmaVanPelt nachsehen
	Aber Low Code Entwicklungsplattformen werden immer besser und sind in vielen Situationen sogar die bessere Option. Welche Gründe sprechen nun für das Entwickeln %nicht nur das Entwickeln auch das Runnen aber wie schreibe ich das?? 
	mit Low Code?
	
	\subsection{Schnittstellenbereitstellung} 
	%weis noch nicht genau
	Lorem ipsum dolor sit amet, consectetuer adipiscing elit.  
	Etiam lobortis facilisissem.  Nullam nec mi et neque pharetra 
	sollicitudin.  Praesent imperdiet mi necante...
	
	\subsection{Lösung auf den Arbeitskräftemangel}
	Durch Low Code können Citizen Developer ohne Know How mit entwickeln. --> Quelle und ausformulieren
	
	Fünfundsechzig Prozent befragter Unternehmen nennen Mangelnde technische Fähigkeiten oder Kenntnisse als Herausforderung in der digitalen Transformation. \cite{EmmaVanPelt.2019} %Figure2
	
	%habe auch die Statistic vom bls in der einleitung die ich hier noch mal aufgreifen könnte
	
	%Anstatt sich stark auf die Programmierung zu verlassen, vereinfachen Low-Code-Plattformen die Anwendungsentwicklung mit Techniken wie Drag-and-Drop-Funktionalität und visueller Anleitung. Das bedeutet, dass jeder in Ihrem Unternehmen, unabhängig von seinen technischen Kenntnissen oder Fähigkeiten, Anwendungen erstellen kann, so dass das Unternehmen einige Aufgaben von der IT-Abteilung übernehmen kann.
	
	%Im Gegensatz zu professionellen Entwicklern kennen sich solche "Citizen Developer" vielleicht nicht so gut mit der manuellen Programmierung aus und haben in der Regel auch keine formale Ausbildung in der Programmierung, aber sie können dennoch Anwendungen mit Low-Code-Plattformen erstellen. Da Low-Code den Prozess der Anwendungserstellung vereinfacht, müssen Bürgerentwickler keine Programmierexperten sein, um effiziente Anwendungen zu erstellen. Durch den Einsatz von Bürgerentwicklern werden auch Ihre IT- und Entwicklungsressourcen entlastet, sodass sie sich auf komplexere Projekte konzentrieren können.
	
	\subsection{verbesserte Zusammenarbeit von Entwicklern und Fachkundigen}
	bessere Zsmarbeit garantiert dadurch auch besser den Kundenanf. gerecht, die Leute die die Geschäftsanf. kennen können auch zumindest zum Teil entwickeln. --> Quelle und ausformulieren
	
	Low-Code-Plattformen verwenden visuelle, deklarative Techniken anstelle von
	Programmierung, um Anwendungen zu erstellen, was Geschäftsexperten ermöglicht, bei der
	Bereitstellung von Lösungen mitzuwirken. \cite{EmmaVanPelt.2019}
	
	% With traditional development practices, feedback loops are like a ferry traveling between two shores. Business users are on one side, and the developers building the application are on the other. Low-code builds a bridge between developers and business users. It makes collaboration easy and encourages clear and frequent input. This helps the end users test assumptions along the way, validating and influencing the application as it takes shape before their eyes.
	
	\subsection{Beschleunigter Entwicklungsprozess}	
	Low-Code beschleunigt die Entwicklung und erfüllt den Bedarf der Unternehmen an Geschwindigkeit. Vierundachtzig Prozent der Unternehmen haben eine Low-Code Entwicklungsplattform oder -werkzeug eingeführt. Diese Firmen sind erfolgreich in ihren Bemühungen bestehende IT-Fähigkeiten zu verbessern, Produkte und Dienstleistungen zu erneuern und agiler zu werden - all das ermöglicht eine schnellere Markteinführung. \cite{EmmaVanPelt.2019}
	
	Vierundsiebzig Prozent befragter Unternehmen nennen die Unfähigkeit so schnell das Softwareprodukt zu liefern wie es das Unternehmen braucht als Herausforderung in der digitalen Transformation. \cite{EmmaVanPelt.2019} %Figure2
	
	\subsection{Reduzierte Entwicklungskosten}
	%With the ability to build more apps in less time, costs decrease. But, that’s not the only driver. Low-code development reduces the need for more developers, reducing hiring costs. And, the right low-code platform can make everyone in the organization—not just IT—more productive.
	
	\subsection{hohe Sicherheitsstandards und geringe Ausfallzeiten} 
	Low-Code-Plattformen können höchste Unternehmensanforderungen erfüllen. Unternehmen mit einer geringsten Toleranz gegenüber Ausfallzeiten und Datenverlusten und Anforderungen an kontinuierliche Audits und unabhängige Sicherheitszertifizierungen, werden ihre Top-Anwendungen am ehesten auf
	Low-Code betreiben. Ihre Befürwortung von Low-Code beweist, dass unternehmenstaugliche 	Low-Code-Lösungen bereits auf dem Markt verfügbar sind.  \cite{EmmaVanPelt.2019}
	
	Einundsechzig Prozent befragter Unternehmen nennen Sicherheitsanforderungen als Herausforderung in der digitalen Transformation. \cite{EmmaVanPelt.2019} %Figure2
	
	%Sicherheitsaspekt wird oft als negativer Punkt betrachtet. Aber auch da kommt es auf die LCDP an, es gibt auch da schon erhebliche Fortschritte, sodass man das mittlerweile oft sogar als Vorteil sehen kann.
	
	%With ever-changing regulations, not to mention their global scale, how can your organization keep up? Low-code development allows for fast change, so you can meet regulatory requirements and stay ahead of deadlines.
	
	\subsection{Schwierigkeiten mit komplexer Geschäftslogik}
	Unternehmen werden sich Low-Code zuwenden, um komplexe Geschäftslogik zu erstellen.
	Während viele Firmen heute benutzerdefinierten Code verwenden, um Anwendungen für komplexe
	Geschäftslogik zu nutzen, sind sie bestrebt, auf dem Erfolg aufzubauen, den die Low-Code-Entwicklung in anderen Bereichen des Unternehmens gebracht hat. In Zukunft werden Unternehmen wahrscheinlich eher Low-Code als benutzerdefinierten Code einsetzen, um geschäftskritischen Anwendungen auszuführen. \cite{EmmaVanPelt.2019}
	
	\subsection{gesteigerte Flexibilität}
	Low-Code-Plattformen mit Top-Anwendungen erweitern Flexibilität, Geschwindigkeit und Automatisierung. Vierundsechzig Prozent der Unternehmen, die Low-Code für Top-Anwendungen nutzen, nennen als Grund für diese Entscheidung weil es die flexibelste Option ist. 
	Mehr als die Hälfte sagen, dass sie Low-Code verwenden, weil es die schnellste Speed-of-Devlivery vorweisen kann. Und 49 \% sagen, dass Low-Code die besten Möglichkeiten zur Prozessautomatisierung bietet. \cite{EmmaVanPelt.2019}
	
	%The downstream effects of increased speed include a better customer experience. With low-code development, organizations can quickly adapt to market changes or customer needs.
	
	%---------------------------------------------------------------------------------------------------
	\section{Fazit und Zukunftsausblick}
	
	Um diese Hausarbeit und die in dieser Hausarbeit, durch unterstützende Literatur,
	herauskristallisierten Informationen und Erkenntnisse abzurunden, wird in diesem Abschnitt
	eine kurze Zusammenfassung, der zuvor erwähnten Informationen und Erkenntnisse,
	geboten. Anhand dieser wird dann eine Beantwortung der Leitfrage „" geboten.
	
	Rückblickend auf die vorhergehenden Kapitel kann man XXX feststellen.
	
	% Forrester ding von emmavanpelt auf seite 7/17 schauen
	
	%Genauere Betrachtung lohnt sich auf jeden Fall!
	
	% Bedarf an Appl wird auch in Zukunft noch viel zu hoch sein. Aber die Entwicklung dieser Apps wird durch LC + NC einfach immer schneller. Wir sind aber gerade noch am Anfang von diesem LC Ansatz.  Ich glaube in Zukunft werden diese DPs immer umfangreicher werden.
	
	Die Verwendung von Low-Code in Unternehmen wird wahrscheinlich zunehmen, da die Plattformen
	Unterstützung für komplexe Geschäftslogik verbessern. Heute würden 30 \% der Unternehmen custom coding nutzen um komplexe Logik zu erstellen, doch dieser Anteil schrumpft auf 16 \%, wenn sie in die Zukunft denken. Stattdessen würden sie es vorziehen, Low-Code für komplexe Geschäftslogik verwenden (30 \%). Dies zeigt uns, dass Unternehmen zwar Vorbehalte gegenüber Low-Code für komplexe Logik haben, aber auch ein Auge auf die zukünftigen Möglichkeiten von Low-Code und
	die Möglichkeit in diese Richtung zu gehen, haben. \cite{EmmaVanPelt.2019}
	
	%noch mal Fazit von Daorsa durchlesen!!
	
	\newpage	
	\printbibliography[title={\section{Referenzen}}]
	
\end{document}
